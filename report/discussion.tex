% $Id: discussion.tex 142 2012-12-22 10:41:32Z danbos $
\chapter{Discussion}
\label{ch:discussion}
\noindent
This was an interesting project to work on. Many of the vulnerabilities are ones I've been aware of for a long time, though others were new. In particular A10 never even occured to me, as its' base assumption (that I want to redirect to an external site through a page on my own) is almost alien to me.

The vulnerability that always surprises me the most is A1, as its' one I've personally been aware of for many years and have been migating in my code since at least 2006.

The largest change from how I've written code in earlier years comes from using Yeoman, Bower, Composer and Grunt, starting the project with not an empty file -- or even a couple of empty files with a Plan --, but with a project of several files to be merged and implemented as I worked. It's also the first project I've been using code sniffers as extensively as I have been here, enforcing PSR-2 coding style on all but a few (in particular the generated classes for database migrations) subsets of classes.

\section{Frameworks}
\label{ch:frameworks}
\noindent
I chose to implement it in Laravel, as re-writing the wheel is more likely to introduce bugs due to misunderstandings about basic demands. That said, when using a framework as the basis for a project, it is very important that one a) understands the code behind it and b) use it as it is intended to be used. If a framework stifles your workflow and you find yourself fighting against it more than it helps, it is not the framework to use.

I have personal experience with both CodeIgniter, Kohana and Laravel, where CI is too compatible with older installations, and breaks the promises of the query string in a way I am not comfortable with. Kohana is nice with a broad range of functions, but in working with it I've often had to work despite it, rather than with it. Laravel is a more comfortable fit as it encourages modularisation (as evidenced by encouraging the use of Composer), doesn't break the promise of the query string functioning despite its' pretty URLs, and the syntactic sugar is pretty.

