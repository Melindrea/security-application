% $Id: implementation.tex 142 2012-12-22 10:41:32Z danbos $
\chapter{Implementation}
\label{ch:implementation}

\section{Technology Stack}
\label{sec:technologies}

\subsection{Frontend}

\subsubsection{CSS}
\noindent
The base typographic styles comes from Typeplate\citep{Typeplate2013}, with headings in Quattrocento Sans and the body type being Libre Baskerville. The intention is to create a more pleasing design as a future step, but it is functional with a max-width of 600px.

The CSS is pre-processed using Sass and Compass, as well as the gem Breakpoint.

\subsubsection{JavaScript}
\noindent
Very little JavaScript is required in this prototype. Modernizr, jQuery and Selectivizr are all concatenated and minified into the head script. The footer script has Magnific, a lightbox script that will be implemented to allow login on the same page.

\subsubsection{HTML}
\noindent
There are two different templating engines in play on the site. The prototype of frontend functionality is built using Assemble.io and Handlebars templates, and Laravel serves Blade templates.

\subsubsection{Grunt}
\noindent
GruntJS is a taskrunner like Make, Rake and similar ones, written in JS. It has several different tasks used for raising the quality of both code and the finished product.

\begin{itemize}
    \item JavaScript tasks: Validates and JShints to ensure that the code style is fulfilled and that there are no syntax errors. Modernizr builds a lean, mean feature detecting machine.
    \item PHP tasks: Lints, sniffs the code and tests using PHPUnit
    \item Build tasks: Assembles pages, concatenates and minifies JS and CSS
    \item Test tasks: Runs Mocha and PHPUnit tests on both the frontend and middle
\end{itemize}

\subsubsection{Bower}
\noindent
Bower is a package manager written in NodeJS, fetching components for the frontend, such as jQuery, Modernizr (for development, not production), Typeplate and Magnific.

\subsection{Middle and backend}
\noindent
The site is written in Laravel, a \enquote{PHP Framework for Web Artisans}\citep{Laravel4}, using a MySQL database. Laravel is a MVC framework, using Blade templating language per default. It is fully PSR-0\citep{PSR} compatible for class loading, and uses Composer as a dependency manager. In fact, cloning the ``Laravel'' project gives a starting point for a project, where you need to install the dependencies using Composer.

The relevant files for the backend are stored in \texttt{/app}. The directory \texttt{/app/src} contains the entirety of the frontend codebase, and any CSS-, image and JavaScript files used by Laravel are compiled into \texttt{/public\_html/assets}. The directory \texttt{/app/storage} needs to be writable by the server, as that is where the cache, logs and views are written to. Sessions are stored in the database.

\subsubsection{Controllers}
\noindent
There are five controllers currently, where only three of them are connected to routes. \texttt{Base} and \texttt{Authorized} are both used as parents, where Base sets up the CSRF-filter to be ran before all \texttt{post}-actions, and Authorized inherits Base, and also sets up actions that should have the \texttt{Auth::check()}-filter ran before them, IE when a user either \emph{must be} or \emph{must not be} logged into the application.

\texttt{Home} is the index page and also the page of policies. It will be further generalized to serve any kind of more ``static'' content, such as what would be referred to as ``pages'' in other contexts.

\texttt{Users} deals with all user actions. Users are seen as a resource in this system, with the login/logout routes added on top of that.

\texttt{Messages} is also a resource, though it only handles listing of messages and storing new messages currently.

\subsubsection{Models}
\noindent
There are two different models used: \texttt{User} and \text{Message}. A third model, \texttt{Base}, sets up validation when an attempt to save the model occurs. The \texttt{User} model has a skeleton of a \texttt{boot} function that will perform actions after a User is created, such as sending an e-mail with login information, and logging the creation.

Every \texttt{User} can have several \texttt{Messages}, which is made easy using Eloquent, the default ORM in Laravel. An expansion on the system will use messages inheriting the base message model to set different types.

\subsubsection{Config}
\noindent
All the configuration options of Laravel are stored as arrays in the \texttt{/app/config} directory. They can be fetched using the syntax \texttt{Config::('filename.keyname')}, and a similar syntax allows you to temporarly set a config option. Another large advantage of Laravel is that by adding new directories under the config directory, one can define different environments, such as testing (pre-defined), production, staging or similar.

\subsection{Server}
\noindent
The current deployment server is a virtual host setup using Vagrant. It has Ubuntu 12.04LTS 32-bits, with MySQL 5.5 and PHP 5.4 under an Apache2 server. Laravel requires Mcrypt as part of its' core encryption functionalities, as well as PHP 5.4+.

\section{Vulnerabilities}
\subsection{A1-Injection}
\noindent
The Database layer uses Eloquent. One of the large advantages to how Laravel is built is that it allows to change what kind of database is used. This system uses MySQL, but it is easy to change it to using SQLite or other drivers. This is doable because it uses PDO and prepared statements. Because it uses those, it is by default more secure against SQL Injection, since the prepared statements will only accept the proper arguments.

For instance, if the \texttt{username} column in a table is a string, and the data supplied from the user is ``test); DROP TABLE users'', that is exactly what the username column will recieve. It will not be interpreted as a command.

\subsection{A2-Broken Authentication and Session Management}
\noindent
As the system is implemented (using Laravel default session management), the session expires when the browser is closed. It will also close the session if the user goes to the route \texttt{users/logout}, after which the user will be redirected back to the main page. However, it is important to know that there is a known feature/bug (per \citep{Google2013cwili} in Chrome where if you have the  setting ``on startup'' set to ``continue where I left off'', the session \emph{will not} be destroyed.

The sessions are stored in the database, in a table called \texttt{sessions}, created by Laravel.

\subsection{A3-Cross-Site Scripting (XSS)}
\noindent
There are two different routes that the system takes to protect against XSS. Fields that should not contain HTML (and which won't be hashed/encrypted) get \texttt{htmlspecialchars} used on them, whereas fields that are expected to contain HTML are stored as-is in the database.

Fields that are expected to contain a certain amount of HTML are then treated right before they're outputted. I have defined an HTML macro (a functionality of Laravel) called \texttt{markdown} to clean and transform the specified string. The macro is called in the blade
template where it is relevant, so the original data is not transformed.

Note that users are discouraged to use HTML in their posts and such, and instead encouraged to use Markdown, a ``text-to-HTML conversion tool for web writers''\citep{Markdown2013}, using the original syntax to be able to work with client side scripts that output a preview of the markdown (the client side scripts are not implemented in this version).

The following steps are taken to protect against XSS:

\begin{enumerate}[1.]
\item
  Using a quite permissive list of allowed HTML elements (see \texttt{app/config/purifier.php}), it cleans the data, in particular removing ``dangerous'' attributes, using the HTMLPurifier, ``a standards-compliant HTML filter library''\citep{Purifier2013}. The only allowed attributes are \texttt{href}, \texttt{class}, \texttt{width}, \texttt{height}, \texttt{alt} and \texttt{src}. Style is forbidden not
  because it is particularly dangerous, but because it risks messing up the design, and all the \texttt{on*} attributes are forbidden as JavaScript should not be allowed for authors. An important HTML element that is of course removed is \texttt{\textless{}script\textgreater{}}, for the same reasons that \texttt{on*} being forbidden.
\item
  PHP Markdown is then used to transform markdown into HTML. This includes some encoding into HTML entities, in particular code contained in codeblocks.
\item
  Final step of the \texttt{HTML::markdown} macro is unrelevant to protecting against XSS, but quite relevant for a more pleasing style, namely using PHP Typography to create ``curly quotes'' and similar things.
\end{enumerate}

\subsection{A4-Insecure Direct Object References}
\noindent
At current there are few places where this would be a concern. The system does not use the traditional definition of directories (being an MVC framework), with the exception of the \texttt{assets} directory. However, all directories are protected using Apache2 from being listed.

In a future implementation, there will be files that can be accessed, but they will all be stored above the public folder and served using routes to protect them from being exposed to the public.

\subsection{A5-Security Misconfiguration}
\noindent
The weakest link at the moment is this one. Laravel supports setting up different configs depending on environment, which will be done before the system goes live but is not yet. One of the key components is that
rather than showing an error trace in production, it will log the errors and report more cleaned-up ones to the user.

The beginnings of ensuring a repeatable development/deployment/testing/production environment has been set up using Vagrant and shell recipes, but not finalized.

\subsection{A6-Sensitive Data Exposure}
\noindent
In this system, the e-mail is protected by being encrypted in the database, and using SSL throughout the more sensitive areas. The e-mail is considered ``sensitive'' as it is the only way to really identify the user, as any other personal information is limited and opt-in only. Another reason can be found in an article by Daniel Cid \citep{Sucuri2013} from 2013 which revisits a hack of PHPBB.com in the beginning of 2009.

The hacker revealed several things, but the key for why I want the e-mails can be found in the following quote:
\begin{quote}
So I login and see what I can come across, wow 400,000 registered emails, I’m sure that will go quick on the black market, sorry people but expect a lot of spam
\end{quote}

\subsection{A7-Missing Function Level Access Control}
\noindent
The current iteration of the system does not have more granulated user roles than logged in/guest, but the principles that are applied currently will apply once it does.

The controller \texttt{Authorized} uses filters defined in Laravel (the details of individual filters can be found in \texttt{app/filters.php}) to determine whether a request should be let through or not. Controllers inheriting \texttt{Authorized} (for instance the \texttt{Users} controller) defines a whitelist of actions that \emph{does not} require a user to be logged in, and may optionally define a guestlist of actions that requires the user to \emph{not} be logged in.

That is, unless a specified action is in the array of whitelisted actions, before the HTTP-request is served, Laravel checks whether the person is logged in. If they are not, they will be redirected to the \texttt{login} route. Once they've logged in, the system will attempt to return them to the page they originally attempted to reach. Examples of whitelisted routes are \texttt{login} and \texttt{users.create}.

On the other hand, if the action is whitelisted and in the guestlist, the system will check that they are not logged in before giving them access to it. After all, why should someone access the login route when they are already logged in?

\subsection{A8-Cross-Site Request Forgery (CSRF)}
\noindent
Another of Laravel's builtin filters is a CSRF-token that is made automatically on each form created using Laravel HTML functions. The \texttt{Base} controller runs a similar filter to the \texttt{Authorized} controller, but where \texttt{Authorized} checks the user capabilities, \texttt{Base} runs the \texttt{csrf} filter on all post-actions. If the token does not match, it throws a TokenMismatchException.

The CSRF-token is also used in the logout process, where the logout links are appended with the token, which is then checked against the value of the user.

\subsection{A9-Using Components with Known Vulnerabilities}
\noindent
To make it easier to keep track of which components are used where and to in the extension be able to write tasks to check everything against at least the more comprehensive databases, all components in PHP use
Composer, and all frontend components use Bower.

One of the development components for PHP is the Security
Checker\citep{SecurityCheck2013} from SensioLabs, which using a gist by Barry vd. Heuvel\citep{Gist2013bvh} was made into a service for Laravel. To check the Composer
components against Sensiolabs database, run \texttt{php artisan security:check}.

Obviously, running automated checks may not catch all vulnerabilities, but they're a good, automated tool to supplement occasional manual checks, and as all components are documented in either \texttt{composer.json} or \texttt{bower.json}, it is easier to track them. There is no mentioning of \texttt{package.json}, as the NPM modules are only used in development, not in a production environment.

\subsection{A10-Unvalidated Redirects and Forwards}
\noindent
Not applicable, as no redirects use data supplied from the user to indicate destination.


