% $Id: implementation.tex 142 2012-12-22 10:41:32Z danbos $
\chapter{Konstruktion}
\label{ch:implementation}
\noindent
Konstruktionsavsnitt ingår ofta i tekniska rapporter, men inte alltid 
i vetenskapliga rapporter.
Här genomför du din analys av problemställningen och formulerar en teknisk 
kravspecifikation.
Du beskriver de viktigaste principerna i de lösningsalternativ som du föreslår, 
utformar och senare i rapporten kommer att utvärdera.
Beskrivningen placeras ibland före, men oftast efter metod- eller 
modellkapitlet.

Tänk på att läsaren sällan är intresserad av alltför detaljerad dokumentation 
av datorprogramkod, algoritmer, kretsscheman, användarhand-ledning, med mera.
Sådana detaljer placeras med fördel i bilagor, om de över huvud taget 
inkluderas.

Under din tidigare universitetsutbildning har du i huvudsak fått små uppgifter 
som vanligen har tagit några minuter eller timmar att lösa.
Ett examensarbete eller en projektkurs kan ibland kännas som en oöverstiglig 
uppgift därför att den är så omfattande, om du inte vet i vilken ände du ska 
börja.
Ett sätt att underlätta stora projekt är att använda \emph{top-down-metoden},
det vill säga successivt dela upp problemet eller konstruktionen i allt mindre 
delproblem eller delsystem, och ange kravspecifikation, problemanalys och 
lösningsförslag för var och en av delarna.
Till slut har du identifierat små och konkreta uppgifter av liknande karaktär 
som du har mött under din tidigare utbildning.

Det är emellertid inte alltid praktiskt möjligt att arbeta enligt 
top-down-metoden, eftersom problemet kan vara för komplext och du inte ser hela 
problembilden från början.
Många gånger måste du kanske växla mellan top-down- och 
\emph{bottom-up-metoden}.
Den senare innebär att du utgår från delar som du redan har, samt från enkla 
problem som du redan vet hur du ska angripa, och som du förväntar sig att du 
kommer att ha nytta av senare under projektet.
Du utökar successivt dessa delar till allt större system och problem, och 
strävar efter att gå i riktning mot projektets mål.

Top-down-metoden har fördelen att den ger rapporten en god struktur, vilket 
underlättar för läsaren.
Dokumentationen följer därför ofta top-down-metoden.
Du kan således dela upp konstruktionsdelen i flera kapitel, och ge dem namn 
efter delproblem och delsystem, exempelvis
\begin{itemize}
	\item Kravspecifikation,
	\item Algoritmer,
	\item Användargränssnitt,
	\item Programdokumentation,
	\item Prototyp, och
	\item Implementering.
\end{itemize}
