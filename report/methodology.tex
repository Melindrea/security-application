% $Id: methodology.tex 142 2012-12-22 10:41:32Z danbos $
\chapter{Metod}
\label{ch:methodology}
\noindent
I examensarbeten på C- och D-nivå räcker det inte att du utför ett praktiskt 
konstruktions- eller programmeringsarbete.
Där måste också en systematisk undersökning genomföras, till exempel en 
utvärdering av den konstruktion du utfört.
Undersökningen bör resultera i objektiva fakta, gärna i form av tabeller och 
diagram, som presenteras i resultatkapitlet.
Ur dessa drar du väl underbyggda egna slutsatser i slutsatskapitlet.
Undersökningen kan vara en jämförelse mellan konkurrerande alternativ eller en 
verifiering av att konstruktionen uppfyller kravspecifikationen.
Du kan låta användare besvara en enkät eller bli intervjuade.
Du kan även utvärdera webbsidor och andra användargränssnitt enligt någon 
allmänt vedertagen förteckning över så kallade användbarhetskriterier.

Metodavsnittet är en redogörelse för ditt metodval och det tillvägagångssätt du 
avser att använda vid undersökningen.
Avsnittet ska inte vara en kronologisk dagbok fylld av ovidkommande detaljer, 
utan det ska beskriva sådant som läsaren måste känna till för att kunna tolka 
dina resultat och återupprepa ditt arbete, exempelvis för att kontrollera 
resultaten.
Här redovisar du verktyg, antaganden, matematiska modeller, prestandamått och 
bedömningskriterier.
Här presenterar du hur du avser att utvärdera och verifiera dina datorprogram 
och tekniska lösningsförslag.
Detta kan innefatta testplan för att kontrollera att konstruktionen fungerar 
och kriterier för att bedöma dess användbarhet.
I forskningsrapporter inom naturvetenskap och teknik heter detta kapitel ofta 
\emph{Modell}, \emph{Systemmodell} eller \emph{Simuleringsmodell}.

I kortare projektrapporter kan metoden vara att genomföra en kritisk 
litteraturstudie.
Då är det särskilt viktigt att ditt arbete resulterar i nya slutsatser som man 
inte kan läsa i annan källa, och att du arbetar målmedvetet, utgående från ett 
klart specificerat problem.

Motivera ditt val av metod eller modell.
Detta val är mycket viktigt, eftersom detta kan sägas vara själva nyckeln till 
resultatet av din undersökning.
Kommentera metodens eventuella svagheter och de problem som kan ha uppstått vid 
själva genomförandet.
Återknyt gärna till problemformuleringen i introduktionskapitlet.
Du kan till exempel skriva ''Problem P1 angrips genom metod M1 och problem P2 
genom \dots''.

I din redogörelse ska du -- beroende på vilken slags rapport det handlar om -- 
finna uppgifter om vad eller vilka du har undersökt och hur du har samlat in 
och bearbetat data.
Eventuella enkäter, intervjufrågor och liknande kan redovisas i ograverad form 
som bilagor, likaså detaljerade beskrivningar av försöksuppställningar, som är 
intressanta endast för den som vill upprepa exakt samma experiment.
