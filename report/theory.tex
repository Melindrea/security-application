% $Id: theory.tex 142 2012-12-22 10:41:32Z danbos $
\chapter{Development of Modern Web Applications}
\label{ch:theory}
\noindent
When developing in a modern context, in particular for the Internet, one needs to
understand how to develop securely, efficiently and without repeating oneself. Though not as much is needed to understand the pertinent areas that this course covers, they are still important to understand the broader strokes.

\section{Separation of Concerns}
\label{sec:architecture}
\noindent
Per \citep{Wikipedia2013ma} Multitier architecture is a ``client–server architecture in which presentation, application processing, and data management functions are logically separated''. In Web Development, that is most commonly generalized to three tiers:

\subsection{Frontend}
\noindent
What is actually served to the browser, whether it's dynamically generated or static. It includes JavaScript, CSS, HTML, typography, graphics, etc. The user interface, or presentation layer. In this system, the least concerns have been put into the frontend, excepting details needed to show understanding of JavaScript-based security issues.

\subsection{Middle}
\noindent
What is most commonly referred to as ``web development'', be it in Ruby on Rails, PHP,
.NET or any other kind of server-side languages. The bulk of this system is written in PHP,
using the framework Laravel.

\subsection{Backend}
\noindent
Where the middle processing technologies generate the frontend consumed by the browser, in most dynamic systems it gets its' information -- data -- from the backend database or data store, whether through a MySQL database, a NoSQL MongoDB or any other kind of system, the backend provides access to the data, and should be properly secured.

\section{oWasp top-10 Vulnerabilities 2013}
\label{sec:owasp2013}
\noindent
oWasp.org \citep{oWasp2013t10} details the 10 most common and/or dangerous vulnerabilities. Most of these require a dynamic system, but having a static HTML site does not remove all the need to think in security.

\subsection{A1-Injection}
\noindent
Injection is still one of the most common vulnerabilities despite the ease with which one can harden ones' system against that kind of intrusion. The most common case is that a system is too trusting, so it allows users to input data without ensuring that it is the right type, or has likely values. A humorous take on the issues comes from this old (in)famous comic \citep{XKCDBobbyTables}, where the mother has named her son ``Robert'); DROP TABLES Students;--'', which led to a school losing its' student table.

When it comes to protecting against SQL injection in particular, the best way is to parametrise the connection and not put data into the database without ensuring that it can at least do no harm.

Injection is mainly protected against in the middle and backend.

\subsection{A2-Broken Authentication and Session Management}
\noindent
The best example I can use here is a site I have worked with on occasion. It is written in PHP and MySQL and is going on ten years by now. Sessions are using only the native PHP drivers, and all the information on the user is sent from page to page in a combination of global variables, session variables and hidden form input(!). To my knowledge, there were at least three bugs related to these, where one I solved, the other two are too closely merged with the rest of the system.

\begin{enumerate}
\item There were two different login systems sharing the same space. One was for admin abilities, the other for basic user abilities. The variable \texttt{\$username} was used in both but different values, so if one logged in as basic user and then did a particular admin action, the \texttt{\$username} was overwritten.
\item Sessions times out at regular intervals, and as the system does not take that into account, whatever was being worked on ended up lost
\item By using FireBug or something similar to change the values on one of the login pages, a person could easily hijack another person's session and items just through manipulating the hidden form input.
\end{enumerate}

Depending on how the authentication and session management is handled, this needs to be dealt with either in the frontend or middle. Most traditional web applications covers this in the middle layer, but if the application is written with a large focus on client side -- IE. EmberJS --, as much care needs to be taken on the frontend.

\subsection{A3-Cross-Site Scripting (XSS)}
\noindent
In a way, XSS is a frontend variant of SQL-injection, as the base problem is the same: An application is too trusting of whatever data it is being supplied, and because of this an attacker might be able to execute scripts on that page that will hijack users in various ways. A very common way is to exploit a comment system, where a very recent instance was with two of the most common cache plugins on WordPress, W3 Total Cache and WP Super Cache \citep{Acunetix2013} and the way they interact with the \texttt{mfunc} function. If the blog allowed HTML in the comment system, it would by default also allow PHP.

\subsection{A4-Insecure Direct Object References}
\noindent
To quote oWASP:

\begin{quote}
A direct object reference occurs when a developer exposes a reference to an internal implementation object, such as a file, directory, or database key.
\end{quote}

Generally, this is a failure in the middle and backend. The server shouldn't allow for directories to be browsed (at the very least, ensure that there is an index.html file in each directory), and configuration should not be accessible publicly. Several very common systems -- PHPBB, MediaWiki, to name two -- by default stores their config file in the root of the \texttt{public_html} folder, that is: the same folder that can be accessed through any browser. Of course, assuming that the config files does not echo anything out, it isn't necessarily that big a risk, unless a misconfiguration of the server leaves the site not recognizing the format, and thus prompting a download.

\subsection{A5-Security Misconfiguration}
\noindent
Security misconfiguration comes in two large areas, where one is easier to deal with than the other, depending on your skill and to a certain extent budget.

\begin{itemize}
\item Framework/Middle configurations are the easiest ones for a web developer to secure. You should always have access to the newest versions of the components you use, and the ability to update them. Ensure that you understand the various options.
\item Server/backend security misconfigurations are reasonably easy to use \emph{if this is your server, or a server you have full access and rights to update software}. If this is shared hosting, you may end up being hit by a vulnerability in the system, such as this recent Ruby on Rails vulnerability\citep{RoR2013Vuln}.
\end{itemize}

\subsection{A6-Sensitive Data Exposure}
\noindent
Any data that is sensitive needs to be protected, both through ensuring that the network can't be sniffed while its' in transit, and ensure that it's at least difficult to use any information stolen through other means, such as by getting a dump of the database.

\subsection{A7-Missing Function Level Access Control}
\noindent
This can be generalised out a bit more than just Access Control, as the same misunderstanding can lead to other problems. The general idea of the misunderstanding is often two-folded:

\begin{enumerate}
\item All users use JavaScript
\item The only way an URL can be found is through a link on the site
\end{enumerate}

Examples of the first leads to, for instance, only validating data using JavaScript, with the assumption that the data will now be correctly validated. Examples of the second leads to admin areas where as long as they know the URL, they can do anything. The better way, obviously, is to validate both clientside and serverside, and to have fail-safes that redirects the user back (or reports a 404) if they're straying.

\subsection{A8-Cross-Site Request Forgery (CSRF)}
\noindent
A CSRF attack is one where the user's browser is tricked/forced into taking an action using its' credentials. As an example, a user posting an image with the source of an URL. As the browser fetches the page, it will also fetch that URL.

The most common -- and quite easy to implement -- is to use a so-called CSRF-token: a unique value sent along with a form submission so that it in the backend can be verified that it is a legitimate request.

\subsection{A9-Using Components with Known Vulnerabilities}
\noindent
A good example of a component with known vulnerabilities -- and a good case on why you should know what you're using! -- is the plugin Timthumb for WordPress (the vulnerability is fixed in the latest versions, but it's important to be aware of this). Back in late 2011, Mark Maunder \citep{MarMaunder2011} had his blog hacked. The short of it is that `Timthumb.php` allows site visitors to load images, and a caching mechanism to not need to re-process images. It will fetch a remote file and put it in a publicly accessible directory, where it was supposed to limit which domains one could fetch images from, but that was part of what was broken.

So, this was 2011. I've already mentioned that it's been fixed, why mention it more? Because there are still themes -- free and paid for -- that includes `timthumb.php` and have not been update. I have personally dealt with a site that was hacked using an inactive theme \emph{with that particular file}.

\subsection{A10-Unvalidated Redirects and Forwards}
\noindent
The usecase I have found on unvalidated redirects and forwards is generally related to various affiliate sites, such as wanting all links to Amazon from your blog to go through a particular page for analytics reasons, and then spammers/phishers find the page and use it for their own nefarious purposes.

A better way to deal with that usecase is to either whitelist which domains are redirected to, or write a system where the url takes parameters that in the backend gets rewritten to the appropriate url.


