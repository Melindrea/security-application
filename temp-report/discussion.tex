% $Id: discussion.tex 142 2012-12-22 10:41:32Z danbos $
\chapter{Diskussion}
\label{ch:discussion}
\noindent
Efter de objektiva resultaten följer kapitlet Diskussion\footnote{%
	Alternativa rubriker är Slutsatser eller Analys.
}, där du presenter dina egna slutsatser, din subjektiva uppfattning, samt 
kritiskt analyserar resultatens tillförlitlighet och generaliserbarhet.

Om denna del är omfattande kan den delas in i flera kapitel eller 
under-kapitel, exempelvis ett analys- eller diskussionskapitel med förklaringar 
till och kritisk granskning av resultaten, ett slutsatskapitel där de 
viktigaste resultaten och slutsatserna presenteras, samt ett avsnitt med 
förslag på fortsatt arbete inom området.

Att återknyta till undersökningens syftes- och målformulering hör till det 
viktigaste i detta kapitel.

Ge gärna utrymme åt svaren på följande frågor:
\begin{itemize}
	\item Vad är projektets nyhetsvärde och viktigaste bidrag till forskningen 
		eller teknikutvecklingen?
	\item Har projektets mål uppnåtts? Har uppdraget utförts?
	\item Vad är svaret på den inledande problemformuleringen?
	\item Har resultatet blivit det väntade?
	\item Är slutsatserna generella, eller gäller de bara under vissa 
		förutsättningar?
	\item Vilken betydelse har metod- och modellvalet för resultaten?
	\item Har nya frågor väckts på grund av resultatet?
\end{itemize}

Den sista frågan inbjuder till möjligheten att ge förslag till andra, 
anknytande undersökningar, det vill säga förslag dels till åtgärder och 
rekommendationer, dels till fortsatt forskning eller utveckling för den som 
vill bygga vidare på ditt arbete.

I tekniska rapporter på uppdrag av företag presenterar du här den 
rekommenderade lösningen på ett problem.
Du kan då göra en konsekvensanalys av lösningen ur tekniskt såväl som 
lekmannaperspektiv, till exempel i fråga om ekonomi, miljö och förändrade 
arbetsrutiner.
Kapitlet innehåller då rekommenderade åtgärder samt förslag på vidare 
utveckling eller forskning, och utgör således beslutsunderlag för 
uppdragsgivaren.
