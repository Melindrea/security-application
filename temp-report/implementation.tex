% $Id: implementation.tex 142 2012-12-22 10:41:32Z danbos $
\chapter{Implementation}
\label{ch:implementation}

\section{Vulnerabilities}
\subsection{A1-Injection}

The Database layer uses Eloquent, the default ORM in Laravel. One of the
large advantages to how Laravel is built is that it allows to change
what kind of database is used. This system uses MySQL, but it is easy to
change it to using SQLite or other drivers. This is doable because it
uses PDO and prepared statements. Because it uses those, it is by
default more secure against SQL Injection, since the prepared statements
will only accept the proper arguments.

For instance, if the \texttt{username} column in a table is a string,
and the data supplied from the user is ``test); DROP TABLE users'', that
is exactly what the username column will recieve. It will not be
interpreted as a command.

\subsection{A2-Broken Authentication and Session Management}

As the system is implemented (using Laravel default session management),
the session expires when the browser is closed. It will also close the
session if the user goes to the route \texttt{users/logout}, after which
the user will be redirected back to the main page. However, it is
important to know that there is
\href{https://productforums.google.com/d/msg/chrome/9l-gKYIUg50/HOvdZbPiuXAJ}{a
known feature/bug} in Chrome where if you have the setting ``on
startup'' set to ``continue where I left off'', the session \emph{will
not} be destroyed.

The sessions are stored in the database, in a table called
\texttt{sessions}, created by laravel.

\subsection{A3-Cross-Site Scripting (XSS)}

There are two different routes that the system takes to protect against
XSS. Fields that should not contain HTML (and which won't be
hashed/encrypted) get \texttt{htmlspecialchars} used on them, whereas
fields that are expected to contain HTML are stored as-is in the
database.

Fields that are expected to contain a certain amount of HTML are then
treated right before they're outputted. I have defined an HTML macro (a
functionality of Laravel) called \texttt{markdown} to clean and
transform the specified string. The macro is called in the blade
template where it is relevant, so the original data is not transformed.

Note that users are discouraged to use HTML in their posts and such, and
instead encouraged to use
\href{http://daringfireball.net/projects/markdown/}{Markdown}, using the
original syntax to be able to work with client side scripts that output
a preview of the markdown (the client side scripts are not implemented
in this version).

The following steps are taken to protect against XSS:

\begin{enumerate}[1.]
\item
  Using a quite permissive list of allowed HTML elements (see
  \texttt{app/config/purifier.php}), it cleans the data, in particular
  removing ``dangerous'' attributes, using the
  \href{http://htmlpurifier.org/}{HTMLPurifier}. The only allowed
  attributes are \texttt{href}, \texttt{class}, \texttt{width},
  \texttt{height}, \texttt{alt} and \texttt{src}. Style is forbidden not
  because it is particularly dangerous, but because it risks messing up
  the design, and all the \texttt{on*} attributes are forbidden as
  JavaScript should not be allowed. An important HTML element that is of
  course removed is \texttt{\textless{}script\textgreater{}}, for the
  same reasons as \texttt{on*} being forbidden.
\item
  \href{http://michelf.ca/projects/php-markdown/}{PHP Markdown} is then
  used to transform markdown into HTML. This includes some encoding into
  HTML entities, in particular code contained in codeblocks.
\item
  Final step of the \texttt{HTML::markdown} macro is unrelevant to
  protecting against XSS, but quite relevant for a more pleasing style,
  namely using \href{https://github.com/adamaveray/PHPTypography}{PHP
  Typography} to create ``curly quotes'' and similar things.
\end{enumerate}

\subsection{A4-Insecure Direct Object References}

At current there are few places where this would be a concern. The
system does not use the traditional definition of directories (being an
MVC framework), with the exception of the \texttt{assets} directory.
However, all directories are protected using Apache2 from being listed.

In a future implementation, there will be files that can be accessed,
but they will all be stored above the public folder and served using
routes to protect them from being exposed to the public.

\subsection{A5-Security Misconfiguration}

The weakest link at the moment is this one. Laravel supports setting up
different configs depending on environment, which will be done before
the system goes live but is not yet. One of the key components is that
rather than showing an error trace in production, it will log the errors
and report more cleaned-up ones to the user.

The beginnings of ensuring a repeatable
development/deployment/testing/production environment has been setup
using Vagrant and shell recipes, but not finalized.

\subsection{A6-Sensitive Data Exposure}

In this system, the e-mail is protected by being encrypted in the
database, and using SSL throughout the more sensitive areas. The e-mail
is considered ``sensitive'' as it is the only way to really identify the
user, as any other personal information is limited and opt-in only.

\subsection{A7-Missing Function Level Access Control}

The current iteration of the system does not have more granulated user
roles than logged in/guest, but the principles that are applied
currently will apply once it does.

The controller \texttt{Authorized} uses filters defined in Laravel (the
details of individual filters can be found in \texttt{app/filters.php})
to determine whether a request should be let through or not. Controllers
inheriting \texttt{Authorized} (for instance the \texttt{Users}
controller) defines a whitelist of actions that \emph{does not} require
a user to be logged in, and may optionally define a guestlist of actions
that requires the user to \emph{not} be logged in.

That is, unless a specified action is in the array of whitelisted
actions, before the HTTP-request is served, Laravel checks whether the
person is logged in. If they are not, they will be redirected to the
\texttt{login} route. Once they've logged in, the system will attempt to
return them to the page they originally attempted to reach. Examples of
whitelisted routes are \texttt{login} and \texttt{users.create}.

On the other hand, if the action is whitelisted and in the guestlist,
the system will check that they are not logged in before giving them
access to it. After all, why should someone access the login route when
they are already logged in?

\subsection{A8-Cross-Site Request Forgery (CSRF)}

Another of Laravel's builtin filters is a CSRF-token that is made
automatically on each form created using Laravel HTML functions. The
\texttt{Base} controller runs a similar filter to the
\texttt{Authorized} controller, but where \texttt{Authorized} checks the
user capabilities, \texttt{Base} runs the \texttt{csrf} filter on all
post-actions. If the token does not match, it throws a
TokenMismatchException.

\subsection{A9-Using Components with Known Vulnerabilities}

To make it easier to keep track of which components are used where and
to in the extension be able to write tasks to check everything against
at least the more comprehensive databases, all components in PHP use
Composer, and all frontend components use Bower.

One of the development components for PHP is the
\href{https://packagist.org/packages/sensiolabs/security-checker}{Security
Checker} from SensioLabs, which using a
\href{https://gist.github.com/barryvdh/6696739}{gist by Barry vd.
Heuvel} was made into a service for laravel. To check the Composer
components against Sensiolabs database, run
\texttt{php artisan security:check}.

Obviously, running automated checks may not catch all vulnerabilities,
but they're a good, automated tool to supplement occasional manual
checks, and as all components are documented in either
\texttt{composer.json} or \texttt{bower.json}, it is easier to track
them. There is no mentioning of \texttt{package.json}, as the NPM
modules are only used in development, not in a production environment.

\subsection{A10-Unvalidated Redirects and Forwards}

Not applicable, as no redirects use data supplied from the user to
indicate destination.


