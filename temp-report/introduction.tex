% $Id: introduction.tex 142 2012-12-22 10:41:32Z danbos $
\chapter{Introduction}
\label{ch:introduction}
\noindent
The 

\section{Bakgrund och problemmotivering}
\label{sec:background}
\noindent
I detta underkapitel ska du snabbt försöka skapa intresse hos läsaren för det 
problemområde du har valt att undersöka.
Visa att du inte bara är insatt i ditt smala tekniska problem, utan att du har 
förståelse för det sammanhang där ditt problem dyker upp, att du kan betrakta 
det ur ett icketekniskt perspektiv och att du känner till den praktiska nyttan 
av den teknik du undersöker eller av den kunskap din studie förväntas ge upphov 
till.

Det är vanligt att den första meningen innehåller en visionär formulering eller 
historisk återblick.
Tänk emellertid på att du inte kan veta säkert hur framtiden kommer att te sig, 
utan bör uttrycka dina visioner på ett nyanserat och sakligt sätt för att 
framstå som trovärdig.

Exempel:
''Mänskligheten har under historien gång \dots
Användandet av Internet och mobiltelefoner har vuxit explosionsartat sedan 
\dots
Nästa steg i utvecklingen förväntas bli \dots
Detta kan leda till problem med \dots
Inom denna studie undersöks om problemet kan lösas med hjälp av \dots
Denna teknologi kan bli särskilt värdefull om några år med tanke på att allt 
fler människor \dots, och på att det finns en växande efterfrågan på marknaden 
efter \dots.''

En teknisk rapport som skrivs på uppdrag av ett företag kan till exempel 
inledas:
''Inom organisationen finns ett ökande behov av \dots och samtidigt växande 
problem med \dots.
Vi har därför fått i uppdrag att genomföra en förstudie om \dots
En lösning på detta problem är angelägen därför att den kan leda till avsevärd 
minskning av kostnader för \dots, ökade marknadsandelar inom \dots samt en 
förbättrad arbetsmiljö.''


\section{Övergripande syfte}
\label{sec:aim}
\noindent
Projektets övergripande syfte är en visionär beskrivning av den riktning 
i vilken du vill arbeta, av vad du hoppas att projektet ska resultera i det 
långa loppet, samt av projektets motiv.
Nyckeln till framgångsrik forskning är ofta att man lyckas formulera en 
intressant frågeställning, syftet blir då att besvara den.
Syftesformuleringen kan vara på hög nivå, det vill säga den behöver inte vara 
klart avgränsad eller konkret.
Det kan vara ett mål som du kanske aldrig kommer att uppnå, eller inte säkert 
kan veta när du har uppnått.
Det kan även vara en problemformulering på hög nivå som inte kan besvaras av 
undersökningens diagram, tabeller eller andra objektiva resultat, men som 
senare kan diskuteras i det avslutande kapitlet.

Exempel:
''Projektets övergripande syfte är att ge upphov till förklaringar till varför 
\dots'',
''Projektets syfte är att jämföra teknik A med teknik B som lösning på behov 
C'',
''Projektets syfte är att identifiera generella principer för sambandet mellan 
X och Y'',
''Projektet syftar till att ge upphov till nya tekniska lösningsförslag inom 
följande problemområde: \dots'',
''Syftet är att ge upphov till ny kunskap inom organisationen om \dots'',
''Projektet syftar till att utgöra ett beslutsunderlag för \dots''.


\section{Avgränsningar}
\label{sec:delimit}
\noindent
Exempel:
''Studien har fokus på \dots
Undersökningen är avgränsad till utvärdering av fall F1 och F2 \dots, 
undersökningens slutsatser bör emellertid vara generellt giltiga för alla \dots
I undersökningen negligeras inverkan av Z, därför att \dots''.


\section{Frågeställning}
\label{sec:problemstatement}
\noindent
Frågeställningen, eller målformuleringen, är en konkretisering av ovanstående 
syftesformulering.
De frågor som specificeras ska besvaras av rapportens resultat, och i dess 
avslutande slutsatser.
Målformuleringen ska vara så konkret att det i efterhand ska gå att avgöra om 
den har uppfyllts, och syftar till att utgöra stoppkriterium för när arbetet är 
slutfört.
Specificera de objektiva numeriska resultat du söker.
Du kan ange vad x- och y-axlarna eller kolumnerna ska visa i de diagram och 
tabeller du har för avsikt att ta fram.

Detta underkapitel skrivs vanligen efter det att du har genomfört teoristudien 
i \prettyref{ch:theory}, och revideras ofta under projektets gång.
Det förekommer att den konkreta problemformuleringen placeras efter 
teoristudien, eftersom det annars kan vara svårt för läsaren att förstå de 
begrepp du använder.
Nackdelen med en sådan disposition är emellertid att läsaren kan tappa 
intresset för ämnet, till följd av att det dröjer så länge innan du som 
författare kommer till kärnpunkten.

Exempel på problemformulering för en vetenskaplig rapport:
''Undersökningen har som mål att besvara följande frågor:
\begin{enumerate}
	\item Vilken betydelse har teknik A i jämförelse med teknik B för 
		prestandamåttet Y vid olika värden på parameter X, för fall F1 och F2?
	\item Vilken vinst ger \dots För matematiska definitioner av X och Y, se 
		modellen i kapitel X.''
\end{enumerate}
I kapitel X specificeras sedan objektivt de numeriska resultaten, till exempel 
vad man kommer att kunna se på x- och y-axlarna i det diagram där man tar 
diskussionen vidare.

Exempel på målformulering för en teknisk rapport:
''Undersökningens mål är att föreslå en lösning på följande tekniska problem: 
\dots
Undersökningen har vidare som mål att verifiera att lösningsförslaget 
tillhandahåller användbara kriterier, samt utvärderar förslaget med avseende på 
prestandamått Y.''


\section{Disposition}
\label{ch:disposition}
\noindent
Beskriv kort rapportens disposition.
Exempel:
''Kapitel X beskriver \dots''.


\section{Författarens bidrag}
\label{ch:contrib}
\noindent
Beskriv vilken del av arbetet som du själv har gjort, och vad du har fått hjälp 
med till exempel av kollegor.
Ange om du har redovisat någon del av arbetet under tidigare kurser eller 
examensarbeten.
Utförs arbetet i grupp kan rapporten redovisa hur ansvaret för arbetets olika 
delar har fördelats mellan författarna.
Givetvis ska alla medförfattare omnämnas för det arbete de lagt ned.
