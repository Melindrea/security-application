% $Id: results.tex 142 2012-12-22 10:41:32Z danbos $
\chapter{Resultat}
\label{ch:results}
\noindent
Resultatkapitlet ingår när du har genomfört en systematisk undersökning, till 
exempel en utvärdering av ett datorprogram som du har utvecklat, vilket krävs 
inom examensarbeten på C- och D-nivå.
I resultatkapitlet redovisas objektiva resultat av en empirisk undersökning, 
exempelvis en sådan utvärdering som nämndes ovan.
Tänk på att eventuella kommentarer i detta kapitel endast får vara av 
förtydligande art.
Dina egna synpunkter och subjektiva\footnote{%
	Det vill säga dina personliga åsikter.
} kommentarer hör hemma i \prettyref{ch:discussion}.

Sträva efter att redovisa resultaten, till exempel enkät-; test-; mät-, 
beräknings- och simuleringsresultat, så överskådligt och lättbegripligt som 
möjligt.
Resultaten presenteras med fördel i diagram- eller tabellform.
Redovisning av intervjuer kan bestå av sammanfattningar, eventuellt 
kompletterade med några konkreta exempel.

Omfattande resultat, till exempel fullständiga sammanställningar av 
enkätresultat, stora tabeller och långa matematiska härledningar, placeras med 
fördel i bilagor.
